\chapter{Methodology}

%---------------------------------------------------------------
\section{The initial project scheduling}

\hspace{15mm}In my project specification, I set my schedule as following:
\begin{itemize}
\item January: Research (2 weeks), design (2 weeks).
\item Febuary: Design (1 week), Implementation (3 weeks).
\item March: Implementation (3 weeks), Testing (1 week).
\item April: Evaluation and report compilation (2 weeks).
\end{itemize}

My supervisor Jo\~{a}o and I met every week or every two weeks to discuss about planning and progresses of my project. 


%---------------------------------------------------------------
\section{The effective approaches}

\hspace{15mm}My initial approach was to become familiar with the embedded device technology. I had to find an adapted, a small and a simple operating system to work with, thereby I chose \textbf{FreeRTOS}\footnote{FreeRTOS is a light-weight Real-Time Operating System.} supported from my supervisor.

My supervisor had already used this system and he had developed applications before. He provided me one of his own device running with FreeRTOS. So, thanks to Jo\~{a}o and his knowledges about this operating system, I had a platform in addition to a massive support from him and the online community to develop my library. As I didn't have any knowledge about FreeRTOS, I read several articles which deals with how it works, and how tasks are scheduled in an real-time way.

As a consequence of this chose and because one the goals of this project is to gain knowledge about low-level development, the library has been build using the C language.

Then, I defined every functionalities of the final applications. Basically, the library's features are simple: it should allow a user to receive and send text statues of a social network. For instance a \textit{message on the wall} in \textbf{Facebook} or a \textit{tweet} in \textbf{Twitter}. Facebook and Twitter are both well known social networks and after some research about them, I choose to build my library suitable for Twitter because of the solid support for \textbf{OAuth} which is a secured protocol to access data. Once again, this choice was supported by Jo\~{a}o.

At this point, I had to defined how to receive and to send tweets, so I've started by looking for any existing solutions. In the next chapter, I will discuss why I chose to build my own library from scratch only using OAuth.

After this key decision, I've learned how OAuth works and how to register an application on Twitter it in order to access to the data.

As I was now aware about what the tools I will use and the way to use them, I designed the library according to the features I wanted to implement.

The next step of the development was to implement the abstract structure of the library. At this stage, I faced lot of issues concerning the use of OAuth, the C language and its requirements. Thanks to my previous years of studies, I learnt the basics of this language but I read a lot of articles, books and tutorials along the implementation.

Once I finished the draft version of the library, I tested every functionalities receiving and sending tweets over my own account and I improved the reliability according to the results of these tests.

% If enought time left : fit this with existing method


%---------------------------------------------------------------
\section{The support tools}

\hspace{15mm}To help myself into the research and the development of the library, I used some additional appropriate tools:
\begin{itemize}
\item A diary: to keep every relevant informations but also as a memory trail of the development chronology.
\item Github\footnote{Github is an online project hosting.}: to back up the source code and share my progress with my supervisor.
\item FreeRTOS POSIX\footnote{Standards to maintaining compatibility between UNIX operating systems.} simulator: to develop and to test my library without any embedded device.
\end{itemize}

\clearpage
