\chapter{Design of the library}


%---------------------------------------------------------------
\section{Functional design}

\subsection{Interactions with Twitter}

\hspace{15mm}Basically, the final library should enable a developer:
\begin{itemize}
\item To authenticate its application to Twitter. 
\item To send tweets to Twitter.
\item To receive tweets from Twitter.
\end{itemize}

And these features have to work whatever the operating system and the hardware architecture.

I chose not implement any storing properties because it is more flexible to let the developer choose the way he wants to store tweets.


\subsection{A user-friendly layer}

\hspace{15mm}This library has to be a user-friendly layer. The developer does not have to know how OAuth works to build its own application to access to Twitter. He just has to give the basic information about the registered application (the public and secret key provided by Twitter) and information about its Twitter account (the login and password).

The first main functionality is the authentication process which gathers all OAuth operations and returns an authentication entity (for instance, typed as a C structure) which could be used by the developer in a later step to send and to receive tweets. This entity contains every required information needed to allow OAuth to connect to Twitter. 

The send functionality is one of the two behaviours which could uses the authentication entity in order to send a tweet on a Twitter profile.

Finally, the receive functionality use the authentication entity as well in order to receive tweets from a Twitter account's \textbf{timeline}\footnote{The timeline is the part of a Twitter profile which contains all tweets sent by a user.}. This functionality includes parsing functions which enables to return a set of tweets entities. 

Each functionality is represented by a single function. Nevertheless, the content of a functionality could be split into several operations each of them represented by another function.



%---------------------------------------------------------------
\section{Implementation design}


\subsection{Functions implementations}

\hspace{15mm}All the steps of the authentication process are gathered into the main authentication function. Basically each use of the OAuth library for a specific stage of the synchronization is surrounded by a set of operations, for this very reason each stage is defined into a distinctive function. As explained above in the chapter \nameref{chap:research}, to authenticate an application to Twitter and then to be able to access the timeline or to send a tweet is simple but requires a few steps:
\begin{itemize}
\item 1- Consumer token: it is the first token given by the Twitter service provider.
\item 2- Request token: it requests the token needed to obtain user authorization.
\item 3- Verifier: it uses the request token in order to get the PIN code (or verifier). 
\item 4- Access token: it uses the verifier to request the final token which will be use to send or receive tweets.
\end{itemize}
This main function gives to the user an authentication entity, that is the one he provides to the behaviour functions (send and receive). This entity is typed as a C structure.

Like every behaviour functions, the send function need the access token to be able to send a text message over Twitter. The main function retrieves the needed information from the authentication entity which are given as parameters in sub-functions\footnote{A sub-function is used by the main function in a distributed way to perform the goal.}. Whatever the sub-function, no field of the authentication entity is directly used, only the main function holds this responsibility.

To get the tweets from the user's timeline, a request is firstly sent to the Twitter service provider. The result is a XML content which is parsed by some sub-functions. These parsing functions determine how many tweets there are in the timeline, and for each of them a new structure is created. Thus, the main function gives to the developer a collection of tweets, each of them is represented by a C structure which contains the most significant information about it (e.g. the date, the text content).


\subsection{Functional diagram of the library}

\begin{figure}[h]
  \centering
  \includegraphics[scale=0.75]{images/functional_design.png}
  \caption{Functional diagram representing the implemented design}
\end{figure}


\clearpage