\chapter{Research}\label{chap:research}

%---------------------------------------------------------------
\section{Operating System: FreeRTOS}

%...........................................
\subsection{Overview}

% Quick overview of the system: Free, open source, GP Licence, light-weight

\hspace{15mm}FreeRTOS is a light-weight and real-time\footnote{A system that should fit to time constraints.} operating system suitable for embedded devices. It is distributed under the GPL\footnote{General public licence is used for free and open source software.} even then with certain exceptions: a developer is required to maintain the kernel as open source, whereas he could remain closed source his applications.

\begin{figure}[h]
  \centering
  \includegraphics[scale=0.125]{images/freertos.jpg}
  \caption{FreeRTOS logo}
\end{figure}

The FreeRTOS source code is tiny and simple, thus it is really easy to port. It is mostly written in C and assembly. The kernel has been ported to around thirty micro-controllers.  


%...........................................
\subsection{Real-Time System}

% Kernel mechanism: priorities and scheduling
\hspace{15mm}One of the main feature provided by FreeRTOS is the way the system performs threads. All tasks are scheduled depending on their priority and sorted according to a round-robin algorithm.

Every files which manage tasks should include the FreeRTOS header \textit{task.h} in order to use the appropriate functions.

A task can be defined as follows and should be of endless loop style:\clearpage
\begin{lstlisting}
void vMyTask (void * pvParameters)
{
	for(;;)
	{
		/* Task code content */
	}
}
\end{lstlisting}

The FreeRTOS thread API\footnote{An Application Programming Interface is a set of routines or functions given by a library.} allows to manage tasks. For instance, to create or delete them:
\begin{lstlisting}
xTaskHandle xHandle;

xTaskCreate( vMyTask, "NAME", STACK_SIZE, &parameters, tskIDLE_PRIORITY, &xHandle );
vTaskDelete( xHandle );
\end{lstlisting}
The \textit{xTaskHandle} element allows to keep a reference to the task that can be then used by other functions

Depending on its presence or its position in the queue, a task can take different states:
\begin{itemize}
\item Ready: the task is ready to run but it is not currently executing because another task is running. From this state, it might become \textit{Suspended} or \textit{Running}.
\item Running: this task is currently executing. From this state, it might become \textit{Blocked}, \textit{Ready} or \textit{Suspended}.
\item Blocked: this task is not available for scheduling until a defined delay period. From this state, it might become \textit{Suspended} or \textit{Ready}.
\item Suspend: this task is not available for scheduling without any timeout. From this state, it might become \textit{Ready}.
\end{itemize}
These states can be changed using the appropriate functions from the API, for instance:
\begin{lstlisting}
/* To suspend a specific task */
vTaskSuspend( xHandle );

/* To resume a suspended task */
vTaskResume( xHandle );

/* To raise a task priority */
vTaskPrioritySet( xHandle, tskIDLE_PRIORITY + 1 );
\end{lstlisting}


%...........................................
\subsection{POSIX simulator}

\hspace{15mm}To develop and test demo tasks in order to get knowledge about FreeRTOS, I worked with a simulator. This is a ported version of the operating system that allows a embedded application to be simulated on a computer running another system.

As I'm a Linux user, so I chose the POSIX simulator. This simulator consist of a set of files representing the kernel and some demo applications which can all be compiled using GCC\footnote{GNU Compiler Collection is a compiler system which can compile several programming languages including C and C++.}.

% Compilation of a library and a task



%---------------------------------------------------------------
\section{Twitter authentication: OAuth protocol}

\subsection{Overview}

(Common authentication mechanism: token, secret key system, include graphic representations)


\subsection{Existing library in C}

(Downloaded and tested library: samples hard to understand, idea: create a simple-to-use library layer)


\subsection{Register an application on Twitter}

(Way and proprieties of the registered application)


\subsection{Required libraries}

(libcurl: overview and it's seem hard to adapt to FreeRTOS, idea: create a very simple HTTP request library)\\
(OpenSSL: overview and it's seem hard to adapt to FreeRTOS, idea)



\clearpage